% \documentclass[fontsize=14pt, paper=a4, pagesize, DIV=calc]{scrreprt}
% разобраться с отступами меж абзацами
% подпись таблиц

\usepackage[
  top=20mm,
  bottom=20mm, 
  left=30mm,
  right=10mm,
  includeheadfoot,
  nohead,
  ]{geometry}


\usepackage{setspace}
\onehalfspacing
\usepackage{enumitem}
% \setlist{topsep=1.5\baselineskip}

% \usepackage{ragged2e}


% \usepackage{polyglossia}
% \setdefaultlanguage{russian} %% устанавливает главный язык документа
% \setotherlanguage{english} %% устанавливает второй язык документа
% \defaultfontfeatures{Ligatures=TeX} %% задаёт свойства шрифтов по умолчанию
% \setmainfont{Times New Roman}
% \setsansfont{Arial} %% задаёт шрифт без засечек
% \setmonofont{Times New Roman} %% задаёт моноширинный шрифт
% doesn't work properly with lipsum, apparently


\usepackage[main=russian, english]{babel}
\babelfont{rm}{Times New Roman}
\babelfont{sf}{Arial}
\babelfont{tt}{FreeMono}

\usepackage{fontspec}
\defaultfontfeatures{Ligatures=TeX}
\newfontfamily\codefont[Scale=MatchLowercase]{Consolas}


\usepackage[utf8]{inputenc}
\usepackage{csquotes}

\setlength{\parindent}{1,25cm}
\setlength{\parskip}{0.4\baselineskip}

% getting that justfied w*rd look
% comment two lines below to get hyphenation back and get rid of infinite amount of overfull warnings
\usepackage[none]{hyphenat}
\sloppy


\usepackage{array} % using this go get a table going for the title page


\renewcaptionname{russian}{\contentsname}{\MakeUppercase{Содержание}}%Table of contents
\renewcaptionname{russian}{\listfigurename}{Список рисунков}%Figures
\renewcaptionname{russian}{\listtablename}{Таблицы}%Tables
\renewcaptionname{russian}{\figurename}{Рисунок} %Figure
\renewcaptionname{russian}{\tablename}{Таблица} %Table
\renewcaptionname{russian}{\lstlistingname}{Листинг}


\setlength{\intextsep}{24pt}


\usepackage{chngcntr}
\counterwithout{section}{chapter}%%%%%%%%%%%%%%%%%%%%%%%%%%%%%%%%%%%%%%%


\usepackage{titlesec}

\titleformat{\chapter}{\normalfont\Large\bfseries\filcenter}{}{0em}{}
\titlespacing*{\chapter}{0pt}{0pt}{24pt}

\titleformat{\section}{\normalfont\bfseries}{\thesection.~}{1em}{}
\titlespacing*{\section}{14pt}{24pt}{4pt}

\titleformat{\subsection}{\normalfont\bfseries}{\thesubsection.~}{1em}{}
\titlespacing*{\subsection}{8pt}{24pt}{4pt}

\titleformat{\subsubsection}{\normalfont\bfseries}{\thesubsubsection.~}{1em}{}
\titlespacing*{\subsubsection}{6pt}{24t}{4pt}

% ----------------------------------------------------------------
\renewcommand{\thefigure}{\arabic{chapter}.\arabic{figure}}
\renewcommand{\thetable}{\arabic{section}.\arabic{table}}
% \usepackage{floatrow}
% \floatsetup[table]{position=topleft}
% ----------------------------------------------------------------
\usepackage{caption}
\DeclareCaptionLabelSeparator{dash}{ -- }% or $\vert$
\captionsetup{
  labelsep=dash,
}
\captionsetup[table]{position=top, justification=raggedright, singlelinecheck=false} % STU TABLE

\usepackage{enumitem}
\setenumerate[1]{label={\arabic*)}} % Global setting


\usepackage{hyperref}
\hypersetup{hidelinks, linktoc=all}


\usepackage[
backend=biber,
bibencoding=utf8,
style=gost-numeric,
language=autobib,
movenames=true,
minnames = 1, 
maxnames = 3, %сократит до первого автора с добавлением [и др.], если число авторов перевышает maxnames
autocite=superscript,
natbib=true
]{biblatex}

\usepackage{etoolbox}
\addbibresource{a.bib}
\usepackage{textcase} % добавляем пакет textcase
%Приведение списка литературы к требованим СФУ
\renewcommand*{\mkgostheading}[1]{{#1}}
\toggletrue{bbx:gostbibliography}%
\renewcommand*{\revsdnamepunct}{\addcomma}%Добавление запятой после фамилии и перед инициалами
\renewcommand*{\labelnamepunct}{\addperiod\space}
% \usepackage[language=english]{lipsum} % just to test if thing are working correctly
\urlstyle{same} % changing url font to the one we use for the rest of the text
\usepackage{graphicx}
\graphicspath{ {./pics/} }
\DeclareGraphicsExtensions{.pdf,.png,.jpg,.svg}
\usepackage{float} % used to make figures actually stay in the place we put them in
% \usepackage[section]{placeins}
\usepackage{scrhack} % stop the compiler from complaining
% \renewcommand{\refname}{Список использованных источников}
\defbibheading{bibliography}{%
    \chapter*{\MakeUppercase{Список использованных источников}}% заголовок на странице
    \addcontentsline{toc}{chapter}{Список использованных источников}% заголовок в содержании
}
%%%%%%%%%%%%%%%%%%%%%%%%%%%%%%%%%%%%%%%%%%%%%%%%%%%%%%%%%%%%%%%%%%%%%%%%%%%%%%%%%%%%%%%%%%%%%%%%%%%%%%%%

\usepackage{listings}
\usepackage{xcolor}

\definecolor{codegreen}{rgb}{0,0.6,0}
\definecolor{codegray}{rgb}{0.5,0.5,0.5}
\definecolor{codepurple}{rgb}{0.58,0,0.82}
\definecolor{backcolour}{rgb}{0.98,0.98,0.95}

\lstdefinestyle{mystyle}{
    backgroundcolor=\color{backcolour},   
    commentstyle=\color{codegreen},
    keywordstyle=\color{magenta},
    numberstyle=\tiny\color{codegray},
    stringstyle=\color{codepurple},
    basicstyle=\ttfamily\footnotesize,
    breakatwhitespace=false,         
    breaklines=true,                 
    captionpos=b,                    
    keepspaces=true,                 
    numbers=left,                    
    numbersep=5pt,                  
    showspaces=false,                
    showstringspaces=false,
    showtabs=false,                  
    tabsize=2
}

\lstset{style=mystyle}



\usepackage{indentfirst}
        
\usepackage{mfirstuc} %first letter capitalizer

\usepackage{titletoc}
  
\titlecontents{chapter}
[0em]                   % adjust left margin
{\rmfamily}             % font formatting
{}{}{\titlerule*[1pc]{.}\contentspage}
% \titleformat{}
